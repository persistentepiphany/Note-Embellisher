\documentclass[12pt,a4paper]{article}
\usepackage{amsmath, amssymb, amsthm}
\usepackage{graphicx}
\usepackage{hyperref}
\usepackage{fancyhdr}
\usepackage[margin=1in]{geometry}
\usepackage{mathptmx}

\title{Calculus Notes}
\author{Test User}
\date{}

\pagestyle{fancy}
\fancyhf{}
\fancyhead[L]{Calculus Notes}
\fancyhead[R]{\thepage}


\newtheorem{theorem}{Theorem}
\begin{document}

\maketitle

\tableofcontents

\section{Introduction to Calculus}

Calculus is a branch of mathematics that studies continuous change. It is divided into two main branches: differential calculus and integral calculus. This document focuses on the basics of differential calculus, particularly derivatives.

\section{Derivatives}

The derivative of a function \( f(x) \) is a measure of how \( f(x) \) changes as \( x \) changes. It is defined as:

\[
f'(x) = \lim_{h \to 0} \frac{f(x+h) - f(x)}{h}
\]

\begin{theorem}
If \( f(x) = x^n \), then \( f'(x) = nx^{n-1} \).
\end{theorem}

\section{Example}

Consider the function \( f(x) = x^2 \). The derivative is calculated as follows:

\begin{itemize}
    \item \( f'(x) = 2x \)
    \item At \( x=3 \), the slope is \( f'(3) = 6 \)
\end{itemize}

\section{Key Properties}

Derivatives have several important properties that simplify calculations:

\begin{enumerate}
    \item \textbf{Linearity:} \((af + bg)' = af' + bg'\)
    \item \textbf{Product Rule:} \((fg)' = f'g + fg'\)
    \item \textbf{Chain Rule:} \((f \circ g)' = (f' \circ g) \cdot g'\)
\end{enumerate}

These properties are fundamental in solving complex calculus problems and are widely used in various applications of mathematics and science.

\end{document}