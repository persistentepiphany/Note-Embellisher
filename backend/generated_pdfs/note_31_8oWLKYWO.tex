\documentclass[12pt,a4paper]{article}
\usepackage{amsmath, amssymb, amsthm}
\usepackage{graphicx}
\usepackage{hyperref}
\usepackage{fancyhdr}
\usepackage[margin=1in]{geometry}
\usepackage{mathptmx}

\title{Note 31}
\author{maximkazakov2005@gmail.com}
\date{}

\pagestyle{fancy}
\fancyhf{}
\fancyhead[L]{MATH21120: Groups and Geometry}
\fancyhead[R]{Week 7 Exercises}
\fancyfoot[C]{\thepage}

\begin{document}

\maketitle

\tableofcontents

\section{Introduction}
Welcome to Week 7 of MATH21120: Groups and Geometry! This week's exercises are designed to deepen your understanding of groups, subgroups, and the concept of cosets. Without a tutorial session, you will have the opportunity to engage with these problems independently. Please remember to document your answers on separate sheets of paper, and feel free to utilize Piazza or the discussion forum for any inquiries you may have. Let's embark on this mathematical journey!

\section{Exercise Set}

\textbf{Note:} For each exercise, you are encouraged to provide detailed solutions and explanations.

\subsection{Examples of Cosets}
For each combination of a group \( G \) and a subgroup \( H \), determine the index \([G : H]\) and list all left cosets of \( H \) in \( G \).

\subsubsection*{(a) \( G = \mathbb{Z}_{15}, \quad H = \langle 12 \rangle \)}
\begin{itemize}
    \item \textbf{Step 1:} Identify the elements of \( H \).
    \item \textbf{Step 2:} Calculate the cosets and the index.
\end{itemize}

\subsubsection*{(b) \( G = \{ \text{id}_4, (1 \, 2), (3 \, 4), (1 \, 2)(3 \, 4) \} \), \( H = \langle (1 \, 2)(3 \, 4) \rangle \)}
\begin{itemize}
    \item \textbf{Step 1:} List the elements of \( H \).
    \item \textbf{Step 2:} Work through the calculation of the index and cosets.
\end{itemize}

\subsubsection*{(c) \( G = \mathbb{R}^* = (\mathbb{R} \setminus \{0\}, \times), \quad H = (\mathbb{R}^+, \times) \)}
\begin{itemize}
    \item \textbf{Step 1:} Define the groups involved.
    \item \textbf{Step 2:} Identify the left cosets and compute the index.
\end{itemize}

\subsection{Example of Left Cosets}
\textbf{Problem Statement:} What are the left cosets of the subgroup \( H = \{ z \in \mathbb{C} \, | \, |z| = 1 \} \) in the group \( \mathbb{C}^* = (\mathbb{C} \setminus \{0\}, \times) \)?

\subsection{Further Exploration of Cosets}
For each pairing of a group \( G \) and a subgroup \( H \), determine the left cosets of \( H \) in \( G \), and find the index \([G : H]\).

\subsubsection*{(a) \( G = \mathbb{R}^*, \quad H = \langle -1 \rangle \)}
\textbf{Discussion:} Analyze the implications of the subgroup.

\subsubsection*{(b) \( G = \mathbb{C}^*, \quad H = \mathbb{R}^* \)}
\begin{itemize}
    \item \textbf{Step 1:} Explore the nature of the cosets.
\end{itemize}

\subsubsection*{(c) \( G = \mathbb{C}^*, \quad H = \mathbb{R}^+ \)}
\textbf{Observation:} Investigate the distinct qualities of the groups.

\subsubsection*{(d) \( G = \mathbb{Z}_{36}, \quad H = \langle 30 \rangle \)}
\textbf{Calculation:} Identify how the index is affected.

\subsection{Theory of Cosets}
\textbf{Theorem Statement:} Let \( G \) be a group, \( H \leq G \), and \( g \in G \). Prove directly (without using lemmas from the notes) that \( gH = H \) if and only if \( g \in H \).

\subsection{Exploration of Group Symmetries}
\textbf{Task:} Find a subgroup of index 2 within the symmetry group of the square. Describe the subgroup and its significance.

\subsection{Left and Right Cosets}
\textbf{Problem:} Provide examples of groups \( G \) with a subgroup \( H \) such that:
\begin{itemize}
    \item \textbf{a)} Left and right cosets differ: \( gH \neq Hg \) for some \( g \in G \).
    \item \textbf{b)} Left and right cosets are the same: \( gH = Hg \) for all \( g \in G \).
\end{itemize}

\subsection{Bijection Between Cosets}
\textbf{Task:} Establish a bijection between the set of left cosets of \( \mathbb{Z} \) in \( \mathbb{R} \) and the set \([0, 1) = \{ x \in \mathbb{R} \, | \, 0 \leq x < 1 \}\). Explain the reasoning behind your mapping.

\subsection{Calculation of Cosets in Symmetric Groups}
Let \( H = \langle (1 \, 2 \, 3 \, 4) \rangle \) be the cyclic subgroup generated by \( (1 \, 2 \, 3 \, 4) \) in the symmetric group \( S_4 \). Compute the following cosets:

\subsubsection*{(a) \( (1 \, 3 \, 4)H \)}

\subsubsection*{(b) \( (2 \, 3)H \)}

\subsubsection*{(c) \( (1 \, 4 \, 3 \, 2)H \)}

\section{Conclusion}
This week’s exercises challenge you to apply theoretical concepts in practical settings. By working through these problems, you will enhance your understanding of group theory, specifically in relation to cosets and subgroup indices. Remember to document your work thoroughly and reach out for assistance when needed. Happy studying!

\end{document}