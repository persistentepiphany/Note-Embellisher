\documentclass[12pt,a4paper]{article}

\usepackage{amsmath, amssymb, amsthm}
\usepackage{graphicx}
\usepackage{hyperref}
\usepackage{fancyhdr}
\usepackage[margin=1in]{geometry}
\usepackage{mathptmx}

\title{A New Project for Me}
\author{8oWLKYWOOJZvjnB4nHxnwhj1wYj2}
\date{}

\begin{document}

\maketitle

\tableofcontents
\newpage

\section{Introduction}

Natural Language Processing (NLP) is a pivotal field within artificial intelligence that enables machines to comprehend and interact with human languages. As we explore the intricacies of NLP, we will delve into language models, their structure, and their applications, offering insights into how these tools facilitate knowledge acquisition and communication. This document aims to lay a foundational understanding of key concepts in NLP, focusing on the mechanisms that power language models and their practical applications.

\section{Outline}

\subsection{NLP for Knowledge Acquisition}
\begin{itemize}
    \item Language Models
    \begin{itemize}
        \item N-gram Character Models
        \item N-gram Word Models
    \end{itemize}
\end{itemize}

\subsection{NLP for Communication}
\begin{itemize}
    \item Grammatical Models
    \begin{itemize}
        \item Lexical Categories
        \item Syntactic Categories
        \item Phrase Structures
    \end{itemize}
    \item Probabilistic Context-Free Grammars (PCFGs)
    \item Parsing Techniques
    \item Ambiguity and Disambiguation
\end{itemize}

\subsection{The Capacity to Process Natural Language}
\begin{itemize}
    \item Distinctions Between Humans and Other Species
    \item The Turing Test
    \item Desirability of NLP Capabilities:
    \begin{itemize}
        \item Information Acquisition
        \item Human Communication
    \end{itemize}
\end{itemize}

\section{Understanding NLP for Knowledge Acquisition}

\subsection{Language Models}

Language models are statistical tools designed to predict the probability distribution of sequences in language, enabling NLP systems to interpret and generate human language effectively. They are essential for carrying out various information-seeking tasks, such as:

\begin{enumerate}
    \item \textbf{Text Classification}
    \begin{itemize}
        \item Assigning predefined categories to text based on its content.
    \end{itemize}
    \item \textbf{Information Retrieval}
    \begin{itemize}
        \item Extracting relevant information from large datasets or documents.
    \end{itemize}
    \item \textbf{Information Extraction}
    \begin{itemize}
        \item Identifying and classifying key elements from text.
    \end{itemize}
\end{enumerate}

\subsection{Language Models and Probability Distributions}

Language can generate an infinite array of sentences, making it impossible to catalog every possible combination. Instead, we utilize probability distributions to assess the likelihood of particular strings or meanings within a language, as illustrated below:

\begin{itemize}
    \item \textbf{Example:} 
    \begin{itemize}
        \item The phrase "Not to be invited is sad" raises questions about its grammatical validity.
        \item Similarly, "He saw a man with a telescope" invites multiple interpretations, demonstrating ambiguity.
    \end{itemize}
\end{itemize}

This complexity underlines the necessity for modeling language as a dynamic system, leading to the construction of probabilistic language models.

\section{Language Models in Detail}

\subsection{N-gram Character Models}

N-gram character models focus on character sequences, calculating the probability of a given character based on the preceding characters in the sequence.

\begin{itemize}
    \item \textbf{Notation:} 
    \begin{itemize}
        \item \( P(c_1:N) \) represents the probability of a sequence of characters.
        \item Example probabilities:
        \begin{itemize}
            \item \( P(\text{'the'}) = 0.027 \)
            \item \( P(\text{'zgq'}) = 0.000000002 \)
        \end{itemize}
    \end{itemize}
    \item \textbf{Characteristics:}
    \begin{itemize}
        \item \textbf{N-grams:} Refers to sequences of n items; for instance:
        \begin{itemize}
            \item 1-gram (unigram)
            \item 2-gram (bigram)
            \item 3-gram (trigram)
        \end{itemize}
        \item \textbf{Markov Chain:} A Markov chain of order \(\eta-1\) implies that the current state \(\eta\) relies solely on a limited number of preceding states (\(\eta-1\) states). The estimation of probabilities is achieved through counting character occurrences within a corpus.
    \end{itemize}
    \item \textbf{Example Calculation:}
    \begin{itemize}
        \item To find the probability of character 'e' following 't' and 'h':
        \[
        P(e|th) = \frac{\text{Count(th)}}{\text{Count(the)}}
        \]
    \end{itemize}
\end{itemize}

A corpus is a structured collection of text documents, which serves as the foundational dataset for analysis.

\subsection{N-gram Word Models}

N-gram word models extend the concept of n-grams from characters to sequences of words. This approach is critical for understanding word relationships and context within sentences.

\begin{itemize}
    \item \textbf{Definitions:}
    \begin{itemize}
        \item \textbf{Bigram:} Two consecutive words (e.g., "please wait").
        \item \textbf{Trigram:} Three consecutive words (e.g., "please wait for").
    \end{itemize}
    \item \textbf{Utilization:}
    \begin{itemize}
        \item The goal is to compute the probability of the next word (w) based on the history of preceding tokens (h).
        \item \textbf{Example Calculation:}
        \begin{itemize}
            \item Given the sequence \( h = \text{"Walden Pond's water is so transparent that,"} \) we want to calculate:
            \[
            P(\text{the}|h) = \frac{\text{Count(h + 'the')}}{\text{Count(h)}}
            \]
        \end{itemize}
    \end{itemize}
    \item \textbf{Challenges:}
    \begin{itemize}
        \item Computing \( P(w_n|w_1:n-1) \) becomes complex due to the rarity of specific contexts, leading to the necessity of approximating probabilities using n-gram models:
        \begin{itemize}
            \item \textbf{Bigram Model:} \( P(w_i|w_{i-1}) \)
            \item \textbf{Trigram Model:} \( P(w_i|w_{i-1}, w_{i-2}) \)
        \end{itemize}
    \end{itemize}
\end{itemize}

\section{Practical Applications of NLP}

NLP techniques have vast implications across various fields, enabling advancements in how we interact with technology.

\subsection{Information Retrieval Systems}

\begin{itemize}
    \item \textbf{Use Case:} Search engines and databases utilize NLP to retrieve relevant documents based on user queries.
    \item \textbf{Implications:} Enhances user experience by providing accurate search results rapidly.
\end{itemize}

\subsection{Chatbots and Virtual Assistants}

\begin{itemize}
    \item \textbf{Use Case:} Applications like Siri and Alexa employ NLP to understand and respond to user commands.
    \item \textbf{Implications:} Revolutionizes customer service and personal assistance, fostering a more interactive user engagement.
\end{itemize}

\subsection{Sentiment Analysis}

\begin{itemize}
    \item \textbf{Use Case:} Businesses analyze customer feedback through sentiment analysis to gauge public opinion.
    \item \textbf{Implications:} Allows companies to adapt their strategies based on consumer sentiment trends.
\end{itemize}

\section{Conclusion}

In conclusion, Natural Language Processing serves as a bridge between human communication and machine understanding. By exploring language models such as n-gram character and word models, we gain insights into the complex and dynamic nature of language. The applications of NLP across various domains highlight its transformative potential in enhancing communication, information retrieval, and understanding of human sentiments. As we continue to advance in this field, the interplay between language and technology will undoubtedly deepen, paving the way for innovations that further integrate AI into our daily lives.

\end{document}